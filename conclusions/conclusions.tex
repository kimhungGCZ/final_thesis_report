\section{Conclusion}
The data volume in mobile networks is booming mostly due to the success of smartphones and tablets. Based on the fact that the mobile Internet traffic will be dominated by the mobile video, the scalability and bandwidth efficiency from multicast routing makes the IP multicast play more important role. However, when considering IP multicast in a wireless mobile environment, it raises several issues such as service disruption, end-to-end delay, packet duplication, non-optimal routing and waste of resource.

To tackle these issues, this thesis proposed the solutions in both PMIPv6 and DMM environments. Through this dissertation, the following objectives are achieved:
\begin{itemize}
\item \textit{Identify the issues and challenges of IP mobile multicast and the evaluation metrics for IP mobile multicast}: In the scope of this thesis, we just highlight such issues as the multicast service disruption, non-optimal routing, end-to-end delay, packet duplication and waste of resource (leave latency) issues. 
\item \textit{Propose an experimental method to achieve the realistic results at a low cost}: The proposed experimental method is a combination of the virtualization and the simulation technique. Based on this study, a PMIPv6 testbed has been implemented. 
\item \textit{Present an effective method for optimizing the service continuity in PMIPv6 and deploy a near-to-real PMIPv6 testbed for IP mobile multicast}: The proposed solution is based on the multicast context transfer and the explicit tracking function allowing the new MAG to obtain the MN's subscription information in advance, thus reducing the multicast service disruption. The testbed allows simulating the mobility of multiple sources and listeners at the same time. Additionally, all modules deployed in the testbed can be used in a real one. 
\item \textit{Propose a load balancing mechanism of multicast flows in PMIPv6}: The proposed solution helps better distribute the load among LMAs to improve the scalability and reliability of the system. 
\item \textit{Introduce a solution for handover of a multihomed node in heterogeneous networks}: Logical interface is used as an abstract layer to hide the change of the physical interface to the IP stack. Thanks to this mechanism, the MN remains unaware of mobility from the multicast service point of view.  
\item \textit{Present an inter-domain mobility support for PMIPv6 networks and a basic support for multicast listener mobility in an inter-domain environment}: The solution allows the data packets to be routed via a near-optimal way by bringing the mobility anchors closer to the MN while the control management can be placed anywhere in the network.
\item \textit{Propose a dynamic multicast mobility anchor (DMMA) mechanism in DMM}: The DMMA not only helps the services to satisfy the strict requirement in terms of service disruption and end-to-end delay, but also offers such benefits as tunnel convergence avoidance, effective tunnel management, route optimization, waste of resource reduction and multicast flow load distribution.
\end{itemize}

\paragraph{Benefit of the Solutions - Application to Real System and Projects}
A part of the dynamic multicast mobility anchor (DMMA) has been implemented in the MEDIEVAL project\footnote{MEDIEVAL project, Homepage: http://www.ict-medieval.eu}. This project aims at providing an architecture to enhance the current mobile Internet and deliver more efficiently mobile video applications. A cross-layer solution has been developed in which two typical services related to multicast are considered i.e., Mobile TV and PBS. Regarding the multicast mobility support, a solution for both multicast listener and source in a DMM environment has been provided. As a part of the overall solution, the multicast mobility module which manages the IP mobility support for the multicast flows has been implemented. In more details, the multicast context transfer and the explicit tracking function are used to accelerate the MN's subscription acquisition process to reduce the service disruption time. For the listener, the multicast packet is always received directly from the multicast infrastructure at the current MAR. For the source, the multicast packet is routed from the current MAR to the anchor one via the mobility tunnel. The real testbed has been deployed to conduct the experiments. The experimental results showed that a small amount of packet loss was observed. Therefore, the session continuity of the video player was possible, with an almost imperceptible handover \cite{ICC_Sergio}.

In the VELCRI project, the solution for handover across heterogeneous networks is one part of the communication system (including Vehicle-to-Grid and Grid-to-Vehicle) to provide the charging service for the EV (Electric Vehicle Charging Services - EVCS). The communication system allows the EV to be always connected to the Smart Grid using different wireless technologies in different phases such as LTE while driving, WLAN while approaching a charging station, and PLC while being docked at a charging station. 

In the SYSTUF project\footnote{SYSTUF project: http://systuf.ifsttar.fr/index-en.php}, the DMMA will be used to provide the multicast service for the users on the public transports e.g., tram and metro. In more details, the goal of the project is to define and implement new services and broadband end-to-end communication system between ground and moving vehicles to improve the quality of urban guided transports. The DMMA will be considered in a high mobility scenario. Also, the mobility predictions can be used to improve the performance of the DMMA. 

\section{Perspectives and Future work}
With the desire to support IP multicast services in a wireless mobile environment, this thesis proposed the solutions for the IP mobile multicast-related issues. However, due to the wide range of the topic defined, several aspects could not be analysed in details, which may potentially be improved. For example, while the focus of this thesis so far has been on the multicast listener mobility, similar idea can be applied for the source mobility. Also, more multicast routing protocols should be investigated e.g., Bidirectional Protocol Independent Multicast (BIDIR-PIM).

Another topic, which would be considered, is the mobility of the node. More mobility models would be applied to study the impact of mobility pattern on the performance of the solution. It can be done by using the existing mobility model in NS-3.

As the proposed solution in Chapter \ref{ch:multicast_dmm} was only validated by the mathematical analysis, a DMM testbed is being deployed using the method described in Chapter \ref{ch:multicast_PMIP}. Additionally, mobility predictions can be used to improve the performance of the DMMA which allows selecting the suitable multicast mobility anchor not only when performing a handover but also at the time the multicast flow is initiated. 

The growing interest in LTE technology by operators brings Multicast/Broadcast Multimedia Service (MBMS) and MBSFN (Multicast/Broadcast over a Single Frequency Network) back to the agenda to support the exponential increase of multimedia distribution services over cellular networks in the next few years. As we do not consider any specific wireless access technology, the IP mobile multicast would be considered in the 3GPP architecture.  
 
In the future, billions of vehicles will be connected to the networks, that creates both new challenges and opportunities for the network operators. Therefore, the DMMA mechanism should be considered, for example, for users on the high-speed vehicles (in the context of NEMO).

Last but not least, we should put our solution in the relation with other technologies e.g., Software Defined Networking (SDN),  Internet of Thing (IoT) and Cloud Computing. For example, the SDN techniques can change mobile core networks and allow for an optimized distributed deployment of virtualized instances of mobile gateways. This could make much more flexible way to process IP packets and flows.  Besides, since IoT applications including Intelligent Transport System (ITS) attract great interests recently, mobility support in IoT is also gaining a lot of momentum. On the other hand, the cloud and the benefits of cloud computing continue to gain significant momentum. Since applications running on clouds are rich media enabled, or collaboration applications, IP multicast can offer benefits to the users as well as to the network operators \cite{cloud_multicast}. Also, sharing the Cloud Computing infrastructure among different network operators also influences the development scenario of DMM \cite{cloud_dmm}.








